\documentclass[12pt,a4paper,openright,twoside]{book}
\usepackage[utf8]{inputenc}

%\newcommand{\thesislang}{italian} % decommentare in caso di tesi in italiano
\newcommand{\thesislang}{english} % commentare in caso di tesi in italiano
\usepackage{thesis-style}

\begin{document}

\pagestyle{empty}
	
\frontmatter

\input{front.tex}


\begin{abstract}	
Max 2000 characters, strict.
\end{abstract}

\begin{dedication} % this is optional
Optional. Max a few lines.
\end{dedication}

\begin{acknowledgements} % this is optional
Optional. Max 1 page.
\end{acknowledgements}

%----------------------------------------------------------------------------------------
\tableofcontents   
\listoffigures     % (optional) comment if empty
\lstlistoflistings % (optional) comment if empty
%----------------------------------------------------------------------------------------

\pagenumbering{arabic}

\mainmatter

%----------------------------------------------------------------------------------------
\chapter{\introductionname}
\label{chap:introduction}
%----------------------------------------------------------------------------------------

Write introduction here.

If you have to cite some source (article, book, etc.),
 use \lstinline|\cite{someref}| where \texttt{someref} is described in \texttt{bibliography.bib}.
%
For instance, here is a citation~\cite{DBLP:journals/eaai/CasadeiVAPD21}.


\section{Motivation}

%
\section{Thesis Structure} % Optional paragraph title
%

(This is an optional paragraph.)
%
Accordingly, the reminder of this thesis is structures as follows.
%
\Cref{chap:background} discusses (briefly describe the content of \cref{chap:background}).
%
Describe other chapters here in a similar way.
%
Finally, \Cref{chap:conclusions} concludes this thesis by summarising its main contribution.

%----------------------------------------------------------------------------------------
\chapter{State of the Art} % or Background
\label{chap:background}
%----------------------------------------------------------------------------------------

Write background here.

This section is likely to contain a lot of citations.
%
For instance in \cite{DBLP:journals/eaai/CasadeiVAPD21} the authors propose a novel means for programming concurrent collective activities.

%----------------------------------------------------------------------------------------
\chapter{Design} % possible chapter for Projects
\label{chap:design}
%----------------------------------------------------------------------------------------

Write design here.

\begin{figure}[h] %h for here, t for top
	\centering
	<\includegraphics[width=0.8\linewidth]{figures/neuron.png}
	\caption{A short description of the picture.}
	\label{fig:classes}
\end{figure}

You may want to reference images in your thesis.
%
In this case, you are encouraged to make them \emph{floating}, and reference them by means of labels.
%
For instance, see \Cref{fig:classes}.
%
%For classes, we describe a class diagram produced by means of \href{http://plantuml.com}{PlantUML}.

%----------------------------------------------------------------------------------------
\chapter{Implementation} % possible chapter for Projects
\label{chap:implementation}
%----------------------------------------------------------------------------------------

Write implementation here.

\lstinputlisting[
	float,
	frame=single,
	language=Java,
	caption={My very first program in Java},
	label={lst:helloworld},
]{listings/HelloWorld.java}

You may need to reference listings in your thesis.
%
In this case, you are encouraged to make them \emph{floating}, and reference them by means of labels.
%
For instance, in \Cref{lst:helloworld}, we describe an hello world program in Java.
%
You may also use listings inline, though.
\begin{lstlisting}[language={java}]
public class Foo { }
\end{lstlisting}

Inline equations as follows: $\int_a^b dx$. Equations as follows

$$
\int_a^b xdx
$$

Numbered equations (and ref via \Cref{eq1} or Equation~\ref{eq1}) as follows

\begin{equation}
\label{eq1}
\int_a^b xdx
\end{equation}

%----------------------------------------------------------------------------------------
\chapter{Validation} % possible chapter for Projects
\label{chap:validation}
%----------------------------------------------------------------------------------------

Write how you validated your work here (e.g., by tests, simulation ...)

%----------------------------------------------------------------------------------------
\chapter{\conclusionsname}
\label{chap:conclusions}
%----------------------------------------------------------------------------------------

Write conclusions here.


%----------------------------------------------------------------------------------------
% BIBLIOGRAPHY
%----------------------------------------------------------------------------------------

%\nocite{*} % uncomment this to show all the reference in the .bib file
\bibliographystyle{plain}
\bibliography{bibliography}

\end{document}
